\documentclass{article}
\usepackage[utf8]{inputenc}
\usepackage[top=1in,bottom=1in,left=1in,right=1in]{geometry}

\title{PS1_Onishi}
\author{Saryu Onishi}
\date{January 2023}

\begin{document}

\maketitle

\section{Introduction}

I am a first-year graduate student doing a Master's in Sports Data Analytics and a PhD in Exercise Physiology in the HES department. I am taking this course because I would like to develop more skills as a data scientist. I am new to this field and initially wanted to take econometrics. However, upon consulting my advisor (Dr. Dan Larson), I was recommended this class. 

I am hoping that through this class, I will become more comfortable working with data and be able to tackle  projects with more data that may require some coding. Although I am considering continuing in academia post-graduation, I prefer an industry job. Hence, I also hope to gain new skills in this class that could make me a more attractive candidate for data science jobs.

As it is my first year, I am still unsure where exactly my research area lies, and I am in the process of figuring out what/how I can use data science in sports. I am specifically interested in endurance sports like triathlon and cycling, and due to the more niche nature of these sports, it has been a little tricky for me to get a grasp of the application of data science in these fields. For my project, I hope to find some interesting data in the field of endurance sports that I could analyze/visualize in some way. I plan on discussing a thesis idea relating to this with my advisor and try to use the class as an opportunity to explore that idea further. 

I have interests in both economics and data science but have formal degrees in neither. I took several economics classes as an undergraduate and loved them (microeconomics, macroeconomics and sports economics). Despite this, I never pursued the path of economics beyond the requirements for my sports management minor. Instead, I find and listen to podcasts and read material that discusses economics in my free time.

\section{Equation}

$a^2+b^2=c^2$

\end{document}
